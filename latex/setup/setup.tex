\documentclass[12pt]{standalone}
\usepackage{graphicx}
\usepackage[version=4]{mhchem}
%% Color package and configuration.
\usepackage[dvipsnames]{xcolor}
\usepackage{tikz}
\usepackage[per-mode=fraction]{siunitx}
\usepackage{pgfplots}
\usepackage{circuitikz}
\usepackage{amsmath,amssymb,xfrac}

%% Commands package.
\usepackage{etoolbox}

%% Fonts package.
\usepackage[no-math]{fontspec}

%%% Configure pgf and tikz.

\pgfplotsset{
    compat=1.18,
    colormap={quanteem}{rgb255=(18, 45, 141) rgb255=(249, 18, 60)}
}

\usetikzlibrary{arrows.meta}
\definecolor{cyan}{HTML}{5BCEFA}
\definecolor{purple}{HTML}{A30262}
\definecolor{green}{HTML}{006b6b}
\definecolor{Blue}{HTML}{091540}

\definecolor{QuanTEEMRed}{HTML}{F9123C}
\definecolor{QuanTEEMBlue}{HTML}{122D8D}
\definecolor{QuanTEEMOrange}{HTML}{FF6700}

%% Color package and configuration.
\usepackage{xcolor}

\begin{document}

\begin{circuitikz}[
    Csm/.style={C, capacitors/scale=1.2},
    Rsm/.style={R, resistors/scale=1.2},
    Zsm/.style={generic, resistors/scale=1.2},
    Lsm/.style={L, inductors/scale=1.2},
    Vsm/.style={american voltage source, sources/scale=1.25},
    Amp/.style={rmeterwa, instruments/scale=1.5, t={\LARGE A}},
    Ampli/.style={amp, blocks/scale=0.75}
]

    \draw[dashed, thick, QuanTEEMRed]
    (-5, 7.7)node[above right] {\LARGE \SI{300}{\kelvin}} rectangle++ (11, 3.8);
    \draw[dashed, thick, QuanTEEMOrange]
    (-5, 5.2)node[above right] {\LARGE \SI{4}{\kelvin}} rectangle++ (11, 2.3);
    \draw[dashed, thick, QuanTEEMBlue]
    (-5, 0.7)node[above right] {\LARGE \SI{10}{\milli\kelvin}} rectangle++ (11, 4.3);

    % GaAs
    \draw[QuanTEEMBlue, fill=QuanTEEMBlue] (-1, 4.5) rectangle++ (2, -2);

    % Ohmic contact
    \draw[thick, fill=black!30] (-0.9, 3.7) rectangle ++ (0.4, -0.4);
    \draw[thick, fill=black!30] (0.9, 3.7) rectangle ++ (-0.4, -0.4);

    \draw [thick] (-0.9, 3.3) --++ (0.4, 0.4) (-0.9, 3.7) --++ (0.4, -0.4);
    \draw [thick] (0.9, 3.3) --++ (-0.4, 0.4) (0.9, 3.7) --++ (-0.4, -0.4);

    \node at (0, 2) {\LARGE Sample};

    % Schottky
    \draw[Goldenrod, fill=Goldenrod] (-0.1, 4.5) rectangle++ (0.2, -0.8);
    \draw[Goldenrod, fill=Goldenrod] (-0.1, 2.5) rectangle++ (0.2, 0.8);

    % RLC.
    \draw (1, 3.5) --++ (1, 0) --++ (0, -0.5)
    to[Csm, l^={\LARGE \(C\)}]++ (0, -1) --++ (0, -0.5) --++ (1, 0)
    (1, 3.5) ++ (1, 0) --++ (2, 0) --++ (0, -0.5) to[Lsm, l^={\LARGE \(L\)}]++ (0, -1)
    --++ (0, -0.5) --++ (-1, 0) node[sground] {};

    % Amplifiers.
    \draw (1, 3.5) ++ (2, 3)
    node[plain mono amp, rotate=90, scale=0.4] (ampA){};
    \node at (2., 6.3) {\LARGE \(A_1\)};

    \draw (1, 3.5) ++ (2, 5)
    node[plain mono amp, rotate=90, scale=0.4] (ampB){};
    \node at (2., 8.3) {\LARGE \(A_2\)};

    \draw (1, 3.5) ++ (2, 0) -- (ampA.in) (ampA.out) -- (ampB.in);

    % Voltages.
    \draw (0, 4.5) --++ (0, 4.5)
    to[Vsm, l^={\LARGE \(V_{\text{qpc}}\)}]++ (0, 1)
    node[sground, rotate=180](G_gate){};

    % \node at (0, 10.8) {Polarisation};

    \draw (-1, 3.5) --++ (-2, 0) --++ (0, 5.5)
    to[Vsm, l^={\LARGE \(V_{\text{bias}}\)}]++ (0, 1)
    node[sground, rotate=180](G_in){};

    % \draw[-Latex] (-4.5, 10.05) node[above]{Injection} --++(2.5, 0);

    % \draw (-1, 3.5) --++ (-2, 0) --++ (0, 5.5);
    % \draw [rounded corners] (-1, 3.5) ++ (-3, 5.5) rectangle++ (2, 2);

    % \draw (-1, 3.5) ++ (-2.75, 5.75) --++ (0.5, 0) ++ (-0.25, 0) --++ (0, 0.5)
    % ++ (0, 0.48) node[esourceshape, rotate=90, scale=1.1] {};

    % \node at (-3.5, 10.2) {\(V(t)\)};

    % \draw[domain=-0.5:0.5, variable=\E, thick]
    % plot ({\E - 2.65}, {0.7*exp(-(\E/0.25)^2) + 9.25});

    % Readout.
    \draw [very thick, purple, rounded corners] (ampB.out) ++ (-1, 0) rectangle++ (2, 2);
    \draw [very thick, purple, -Latex] (2.25, 9.2) --++ (0, 1.5) node[right] {\(PSD\)};
    \draw [very thick, purple, -Latex] (2.25, 9.2) --++ (1.5, 0) node[above] {\(f\)};

    % \draw[-Latex] (6, 10.05) node[above]{Measurement} --++(-2, 0);

    \draw [very thick, purple, domain=0.05:1.38, variable=\E]
    plot ({\E + 2.25}, {0.05/\E + 9.3 + 0.15*exp(-((\E - 0.9)/0.2)^2)});

\end{circuitikz}
\end{document}
