\documentclass[12pt]{standalone}
\usepackage{graphicx}
\usepackage[version=4]{mhchem}
%% Color package and configuration.
\usepackage[dvipsnames]{xcolor}
\usepackage{tikz}
\usepackage[per-mode=fraction]{siunitx}
\usepackage{pgfplots}
\usepackage{circuitikz}
\usepackage{amsmath,amssymb,xfrac}

%% Commands package.
\usepackage{etoolbox}

%% Fonts package.
\usepackage[no-math]{fontspec}

%%% Configure pgf and tikz.

\pgfplotsset{
    compat=1.18,
    colormap={quanteem}{rgb255=(18, 45, 141) rgb255=(249, 18, 60)}
}


\usetikzlibrary{arrows.meta}
\definecolor{cyan}{HTML}{5BCEFA}
\definecolor{purple}{HTML}{A30262}
\definecolor{green}{HTML}{006b6b}
\definecolor{Blue}{HTML}{091540}

\definecolor{QuanTEEMRed}{HTML}{F9123C}
\definecolor{QuanTEEMBlue}{HTML}{122D8D}
\definecolor{QuanTEEMOrange}{HTML}{FF6700}

%% Color package and configuration.
\usepackage{xcolor}

\begin{document}

\begin{circuitikz}[
    Csm/.style={C, capacitors/scale=0.7},
    Rsm/.style={R, resistors/scale=0.7},
    Lsm/.style={L, inductors/scale=0.7},
    Vsm/.style={sinusoidal voltage source, sources/scale=1.25}
]
    \draw [black, fill=QuanTEEMBlue] (0, -2) rectangle (5.5, 0);

    \draw[-Latex, very thick] (0, -2) -- (6, -2) node[below, midway] {\Large $t$};
    \draw[-Latex, very thick] (0, -2) -- (0,2) node[above] {\Large $\left|\psi(t)\right|^2$};

    \draw[-Latex] (2.1,0.75) --++ (1.3,0);

    \node at (3, 1.8) {\Large $\psi(t) \propto \frac{1}{t + i W}$};

    \draw[domain=0:5.5, samples=1000,variable=\x, black, thick, fill=cyan]
    plot ({\x},{1.5 * 0.03/(0.03 + (\x - 1.7)^2)});

    \draw (0, 0) to[short, *-]++ (-2, 0) to[Vsm, l_={\Large \(V(t)\)}]++ (0, -2) node[sground] {};

    % \node at (1.5, -3.5) {\Large $V(t) = \frac{\hbar}{e} \frac{2 W}{t^2 + W^2}$};

    \draw [black, fill=QuanTEEMBlue] (6.5, -2) rectangle (8.5, 0);

    \draw[domain=6.5:8.5, samples=100,variable=\x, black, thick, fill=cyan]
    plot ({\x}, {1.3*exp(-2.*(\x - 6.5))}) --++ (-2, 0);

    \draw[-Latex, very thick] (6.5, 0) -- (9, 0) node[above] {\Large $\left|\psi(E)\right|^2$};
    \draw[-Latex, very thick] (6.5, -2) -- (6.5,2) node[above] {\Large $E$};
    \node at (6, 0) {\Large $E_f$};

    \draw (10.2, 1) --++ (0.1, 0) --++ (0, -0.8) --++ (-0.1, 0);
    \draw (10.2, -0.2) --++ (0.1, 0) --++ (0, -1.8) --++ (-0.1, 0);

    \draw (10.3, 0.6) --++ (0.1, 0) node [right] {\Large Leviton};
    \draw (10.3, -1.1) --++ (0.1, 0) node [right] {\Large Fermi sea};
    \draw [opacity=0] (12.3, -1.1) --++ (1, 0);

\end{circuitikz}
\end{document}
