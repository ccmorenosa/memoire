\documentclass[12pt]{standalone}
\usepackage{graphicx}
\usepackage[version=4]{mhchem}
%% Color package and configuration.
\usepackage[dvipsnames]{xcolor}
\usepackage{tikz}
\usepackage[per-mode=fraction]{siunitx}
\usepackage{pgfplots}
\usepackage{circuitikz}
\usepackage{amsmath,amssymb,xfrac}

%% Commands package.
\usepackage{etoolbox}

%% Fonts package.
\usepackage[no-math]{fontspec}

%%% Configure pgf and tikz.

\pgfplotsset{
    compat=1.18,
    colormap={quanteem}{rgb255=(18, 45, 141) rgb255=(249, 18, 60)}
}

\usetikzlibrary{arrows,shapes,positioning,shadows,trees,3d,calc}

\usetikzlibrary{arrows.meta}
\definecolor{cyan}{HTML}{5BCEFA}
\definecolor{purple}{HTML}{A30262}
\definecolor{green}{HTML}{006b6b}
\definecolor{Blue}{HTML}{091540}

\definecolor{QuanTEEMRed}{HTML}{F9123C}
\definecolor{QuanTEEMBlue}{HTML}{122D8D}
\definecolor{QuanTEEMOrange}{HTML}{FF6700}

%% Color package and configuration.
\usepackage{xcolor}

\begin{document}

\begin{circuitikz}[
    Csm/.style={C, capacitors/scale=0.7},
    Rsm/.style={R, resistors/scale=0.7},
    Lsm/.style={L, inductors/scale=0.7},
    Vsm/.style={american voltage source, sources/scale=1.25},
    Amp/.style={rmeterwa, instruments/scale=1.5, t={\LARGE A}}
]

    % GaAs
    \draw[thick, fill=QuanTEEMBlue!50] (-3, 2.25, 3) --++
    (6, 0, 0) --++ (0, -2.25, 0) --++ (-6, 0, 0) --++ (0, 2.25, 0) --++ (0,
    0, -6) --++ (6, 0, 0)--++ (0, -2.25, 0)  --++ (0, 0, 6);
    \draw[thick] (3, 2.25, 3) --++ (0, 0, -6);

    % % 2DEG
    \draw[fill=cyan] (-3, 2.5, 3) --++ (6, 0, 0) --++ (0,
    -0.25, 0) --++ (-6, 0, 0) --++ (0, 0.25, 0) --++ (0, 0, -6) --++ (6, 0,
    0)--++ (0, -0.25, 0)  --++ (0, 0, 6);
    \draw (3, 2.5, 3) --++ (0, 0, -6);

    % Pinch
    \begin{scope}[canvas is xz plane at y=2.5]

        \draw[fill=black!75] (-0.6, 3) --++ (1.2, 0) --++ (0, -2)
        to[in=270, out=270]++ (-1.2, 0) --++ (0, 2);

        \draw[fill=black!75] (-0.6, -1) to[in=90, out=90]++ (1.2, 0)
        --++ (0, -2) --++ (-1.2, 0) --++ (0, 2);

    \end{scope}
    \begin{scope}[canvas is xy plane at z=3]

        \draw[fill=black!75] (-0.6, 2.5) rectangle++ (1.2, -0.25);

    \end{scope}

    % Ohmic contact
    \draw[fill=black!30] (-2.75, 3.5, 0.5) --++ (1.0, 0, 0) --++ (0, -1.0,
    0) --++ (-1.0, 0, 0) --++ (0, 1.0, 0) --++ (0, 0, -1.0) --++ (1.0, 0,
    0)--++ (0, -1.0, 0)  --++ (0, 0, 1.0);
    \draw (-1.75, 3.5, 0.5) --++ (0, 0, -1.0);

    \draw[fill=black!30] (1.75, 3.5, 0.5) --++ (1.0, 0, 0) --++ (0, -1.0,
    0) --++ (-1.0, 0, 0) --++ (0, 1.0, 0) --++ (0, 0, -1.0) --++ (1.0, 0,
    0)--++ (0, -1.0, 0)  --++ (0, 0, 1.0);
    \draw (2.75, 3.5, 0.5) --++ (0, 0, -1.0);

    % AlGaAs
    \draw[thick, fill=gray!50, opacity=0.35] (-3, 3.5, 3) --++ (6, 0, 0)
    --++ (0, -1.25, 0) --++ (-6, 0, 0) --++ (0, 1.25, 0) --++ (0, 0, -6) --++
    (6, 0, 0)--++ (0, -1.25, 0)  --++ (0, 0, 6);
    \draw[thick, opacity=0.35] (3, 3.5, 3) --++ (0, 0, -6);

    % \draw[thick, fill=gray!90, opacity=0.35] (-3, 3.5, 3) --++ (6, 0, 0)
    % --++ (0, -0.5, 0) --++ (-6, 0, 0) --++ (0, 0.5, 0) --++ (0, 0, -6) --++
    % (6, 0, 0)--++ (0, -0.5, 0)  --++ (0, 0, 6);
    % \draw[thick, opacity=0.35] (3, 3.5, 3) --++ (0, 0, -6);

    % Schottky
    \draw[fill=Goldenrod] (-0.5, 3.75, 3) --++ (1, 0, 0) --++
    (0, -0.25, 0) --++ (-1, 0, 0) --++ (0, 0.25, 0) --++ (0, 0, -2) --++
    (1, 0, 0) --++ (0, -0.25, 0)  --++ (0, 0, 2);
    \draw(0.5, 3.75, 3) --++ (0, 0, -2);

    \draw[fill=Goldenrod] (-0.5, 3.75, -1) --++ (1, 0, 0) --++
    (0, -0.25, 0) --++ (-1, 0, 0) --++ (0, 0.25, 0) --++ (0, 0, -2) --++
    (1, 0, 0) --++ (0, -0.25, 0)  --++ (0, 0, 2);
    \draw(0.5, 3.75, -3) --++ (0, 0, 2);

    % % Circuit
    % \draw[thick] (-2.75, 3., 0.5) --++ (-2, 0, 0) node[above left=0.9cm]
    % {\LARGE \(V_{\text{bias}}\)} to[Vsm] ++ (0, 2, 0)
    % node[sground, rotate=180]{};

    % \draw[thick] (0, 3.75, -3) --++ (0, 0, -2)
    % node [above right=1cm] {\LARGE \(V_{\text{qpc}}\)} to[Vsm] ++ (3, 0, 0)
    % node[sground, rotate=90]{};

    % \draw[thick] (2.75, 3., 0) --++ (2, 0, 0)
    % to[Amp, l_={\LARGE Signal}]++ (2, 0, 0) node[sground, rotate=90]{};

    % % Labels
    \draw[Latex-,opacity=1] (-3.0, 0.625, 3) --++ (-1, 0, 0) node[left] {\LARGE \ce{GaAs}};
    \draw[Latex-,opacity=1] (-3.0, 2.25, 3) --++ (-1, 0, 0) node[left] {\LARGE 2DEG};
    \draw[Latex-,opacity=1] (-3.0, 3.5, 3) --++ (-1, 0, 0) node[left] {\LARGE \ce{AlGaAs}};
    \draw[Latex-,opacity=0] (-3.0, 3.5, -3) --++ (-1, 0, 0) node[left] {\LARGE \(n-\ce{AlGaAs}\)};
    \draw[Latex-,opacity=1] (-2.5, 3.5, 0.0) --++ (0, 1.5, 0) node[above] {\LARGE Ohmic Contact};
    \draw[Latex-,opacity=1] (0, 3.75, -2) --++ (0, 1.5, 0) node[above] {\LARGE Schottky};

\end{circuitikz}
\end{document}
