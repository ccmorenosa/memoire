\documentclass[12pt]{standalone}
\usepackage{graphicx}
\usepackage[version=4]{mhchem}
%% Color package and configuration.
\usepackage[dvipsnames]{xcolor}
\usepackage{tikz}
\usepackage[per-mode=fraction]{siunitx}
\usepackage{pgfplots}
\usepackage{circuitikz}
\usepackage{amsmath,amssymb,xfrac}

%% Commands package.
\usepackage{etoolbox}

%% Fonts package.
\usepackage[no-math]{fontspec}

%%% Configure pgf and tikz.

\pgfplotsset{
    compat=1.18,
    colormap={quanteem}{rgb255=(18, 45, 141) rgb255=(249, 18, 60)}
}

\usetikzlibrary{arrows.meta}
\definecolor{cyan}{HTML}{5BCEFA}
\definecolor{purple}{HTML}{A30262}
\definecolor{green}{HTML}{006b6b}
\definecolor{Blue}{HTML}{091540}

\definecolor{QuanTEEMRed}{HTML}{F9123C}
\definecolor{QuanTEEMBlue}{HTML}{122D8D}
\definecolor{QuanTEEMOrange}{HTML}{FF6700}

\usetikzlibrary{arrows,shapes,positioning,shadows,trees,3d,calc, math}

%% Color package and configuration.
\usepackage{xcolor}

\begin{document}

\begin{circuitikz}[
    Csm/.style={C, capacitors/scale=0.7},
    Rsm/.style={R, resistors/scale=0.7},
    Lsm/.style={L, inductors/scale=0.7},
    Vsm/.style={sinusoidal voltage source, sources/scale=1.25}
]

    \tikzmath{
        \xFS = 0; \dxFS = 4;
        \yFS = 0; \dyFS = 3;
    }

    \draw [black, fill=QuanTEEMBlue] (\xFS, \yFS) rectangle++ (\dxFS, -\dyFS);

    \draw[-Latex, very thick] (0, -3) -- (0, 3) node[above] {\Large $E$};

    \draw (0, 0) to[short, *-]++ (-2, 0) to[Vsm, l_={\Large \(V(t)\)}]++ (0, -2) node[sground] {};

    \draw [-Latex, thick] (1.5, 0.5) --++ (2, 0);

    % \node at (-2, 2) {\Large \(Q = 1\)};

    \draw [black, fill=cyan] (1, 0.5) circle (5pt);
    \draw [black, fill=cyan] (1, 1) circle (5pt);
    \draw [black, fill=cyan] (1, 1.5) circle (5pt);
    \draw [black, fill=cyan] (1, 2) circle (5pt);
    % \draw [black, fill=cyan] (1, 2.5) circle (5pt);
    \node [rotate=90] at (1, 2.5) {\large \(\cdots\)};

    \draw [black, fill=white] (1, -0.5) circle (5pt);
    \draw [black, fill=white] (1, -1) circle (5pt);
    \draw [black, fill=white] (1, -1.5) circle (5pt);
    \draw [black, fill=white] (1, -2) circle (5pt);
    % \draw [black, fill=white] (1, -2.5) circle (5pt);
    \node [rotate=90, white] at (1, -2.5) {\large \(\cdots\)};

    \draw (\xFS + \dxFS + 0.2, \yFS - \dyFS) --++
    (0.1, 0) --++ (0, \dyFS) --++ (-0.1, 0);
    \draw (\xFS + \dxFS + 0.3, \yFS - \dyFS/2) --++
    (0.1, 0) node [above, rotate=-90] {\Large Fermi sea};

    \draw [opacity=0] (\xFS + \dxFS, \yFS - \dyFS) --++ (0, -0.3);

\end{circuitikz}
\end{document}
